\documentclass[a4paper]{article}

\usepackage[english]{babel}
\usepackage[utf8]{inputenc}
\usepackage{amsmath}
\usepackage{graphicx}
\usepackage{amssymb}
\usepackage{amsthm}
\usepackage{tikz-cd}
\usepackage{mathrsfs}
\usepackage[colorinlistoftodos]{todonotes}
\usepackage{enumitem}
\usepackage{yfonts}

\title{Toric varieties I}

\author{Yang Li}

\date{\today}
\newtheorem{thm}{Theorem}[section]
\newtheorem{lem}[thm]{Lemma}

\newtheorem{defn}[thm]{Definition}
\newtheorem{eg}[thm]{Example}
\newtheorem{ex}[thm]{Exercise}
\newtheorem{conj}[thm]{Conjecture}
\newtheorem{cor}[thm]{Corollary}
\newtheorem{claim}[thm]{Claim}
\newtheorem{rmk}[thm]{Remark}

\newcommand{\ie}{\emph{i.e.} }
\newcommand{\cf}{\emph{cf.} }
\newcommand{\into}{\hookrightarrow}
\newcommand{\dirac}{\slashed{\partial}}
\newcommand{\R}{\mathbb{R}}
\newcommand{\C}{\mathbb{C}}
\newcommand{\Z}{\mathbb{Z}}
\newcommand{\N}{\mathbb{N}}
\newcommand{\LieT}{\mathfrak{t}}
\newcommand{\T}{\mathbb{T}}


\begin{document}
	\maketitle
	
	\begin{abstract}
		The lecture notes are based on Tom Coates' lecture on toric varieties. A few references:
		
		\begin{itemize}
			\item M. Audin, Toric actions on symplectic manifolds
			\item W. Fulton, Toric varieties
			\item Cox-Schenck-Little, Toric varieties
		\end{itemize}
	\end{abstract}
	
	\section{Introduction}
	
	
	Aim of the lecture: Sketch several viewpoints on toric manifolds.
	
	\begin{itemize}
		\item Symplectic viewpoint: X compact symplectic 2n-D manifold, with a torus action of $\T$ on X, where the torus is the real n-D torus and the action is Hamiltonian and effective.
		\item Algebraic viewpoint: smooth projective variety, with a dense open subset $T_{\C} \in X$, such that $T_{\C}$ acts on X extending the left action on the open dense copy of $T_{\C}$.
		\item Symplectic viewpoint': The symplectic quotient of $\C^{n}$ by the $(S^{1})^{k}$ action, where $(S^{1})^{k}$ is a subgroup of $(S^{1})^{n}$ and acts by restricting the usual linear action of $(S^{1})^{n}$ on $\C^{n}$.
		\item algebraic viewpoint': the GIT quotient of $\C^{n}$ by the $(\C^{*})^{k}$ action, where $(\C^{*})^{k}$ acts by restricting the usual linear action of $(\C^{*})^{n}$ on $\C^{n}$.
		
	\end{itemize}
	
	
	The subject is attractive because it provides good test ground for general theories; the quantities are very computable and admit explicit combinatorial descriptions. There are also many interesting examples such as the quintic 3-folds.
	
	\section{Symplectic geometry and polytopes}
	
	\subsection{moment map and torus fibration}
	Let  X=2n-dim symplectic manifold, $\T$=n-dim real torus.
	T acts on X effectively through Hamiltonian action.  Let $\LieT$=Lie(T) be the Lie algebra of $\T$. There is a (moment map) $$\mu\colon M\to \LieT^\ast,$$ which is T-equivariant, such that for each $v\in\mathrm{Lie}(\mathbb{T})$ we have $$\omega(v^\sharp,\cdot)\cong \mathrm{d}<\mu,v>,$$ 
	
	Notice $\LieT$ is noncanonically isomorphic to $\R^{n}$.
	But there is a canonical sublattice $\Lambda=ker (exp)$, and the dual lattice sits inside the dual space.
	$$\mathrm{Lie}(\mathbb{T})\cong \mathbb{R}^n\cong \Lambda\otimes_{\mathbb{Z}}\mathbb{R}$$
	
	We are regarding $\LieT^{*}$ as an affine manifold with an integral structure.
	
	Good local picture:
	T acts on X effectively, so for a generic $x \in X$ the T-orbit of X is a copy of T. Locally near x, the moment map is a fibration:
	$$\LieT^* \times \T \mapsto \LieT^*$$
	
	\begin{rmk}
		This is in fact a special case of coadjoint actions.
	\end{rmk}
	A better description taking into account the integral structure: \\
	
	$\T=Lie(T)/\Lambda$, 
	So setting $V=\LieT^{*}$, locally we get $V \times V^{*}/{\Lambda}^{*} \mapsto V$  
	with the symplectic structure on X (locally) given by the canonical symplectic form on the cotangent bundle  $T^{*}V$.
	
	
	
	So the torus fibres are Lagrangian.
	
	\subsection{Digression: the link with Hamiltonian mechanics}
	
	A Hamiltonian system is a triple $ (X, \omega, f)$ where $\omega$ is the symplectic form and f is a function. Then f induces a flow via its Hamiltonian vector field.\\
	
	Given any other observable g (\ie, functions from X to $\R$), the value of g changes under the evolution according to the rule: $$\frac{dg}{dt}=\{f,g\}$$
	
	In particular, if f and g Poisson commute then g is a conserved quantity. In classical mechanics conserved quantities come from symmetries of the system and they simplify the problem by reducing the degree of freedom. 
	In general one can have at most n independent Poisson commuting functions. Then Arnold-Liouville theorem roughly says that if you achieve this maximal number of conserved quantities $\{f_i\}$, you can find some coordinate system which puts the flow into an exceptionally simple form:
	$$ \omega=\sum_i{df_i \wedge dy_i}$$
	and the evolution of the system is linear in the y coordinates.\\
	
	This coordinate system is called the action-angle variables. This is another way to understand the Lagrangian toric fibration. The Hamiltonian flows naturally give torus group actions, if we are given extra compactness conditions.
	
	
	\subsection{Compactification and polytopes}
	
	Local model: on a big open $X^{0} \in X$ we have $\R^n \times (S^1)^n \mapsto \R^n$ via $\mu$. The action-angle coordinates are essentially just polar coordinates on $\C^*$. \\
	
	
	Question: How to compactify our local picture?\\
	
	Answer: the information you need is encoded by a polytope.
	
	
	\begin{thm}
		\rm (Atiyah-Guillemin-Sternberg) The image of the moment map is a convex polytope. In fact, it is the convex hull of the images of the fixed points of the torus action. 
	\end{thm}
	
	This means we get a polytope from the toric manifold.
	
	\begin{eg}
		$X=\C\mathbb{P}^1$, $T=(S^1)^n $ acts on X by:
		$$ (exp(i\theta_1, \ldots, exp(i\theta_n)).[z_0,\ldots, z_n]=[z_0, exp(i\theta_1)z_1, \ldots]$$
		the moment map is 
		$$\mu ([z_0,\ldots, z_n])=-\frac{1}{\sum_i{|z_i|^2}}(|z_1|^2, \ldots)$$
		so the image of $\mu$ is the convex hull of (0,0), (-1,0), (0,-1).
	\end{eg}
	
	\begin{rmk}
		The moment map is really only defined up to a constant shift.
	\end{rmk}
	
	
	\begin{defn} A \it Delzant polytope \rm in $\Lambda\otimes_{\mathbb{Z}}\mathbb{R}$ is a convex polytope such that\\
		(i) the vertices are in $\Lambda$,\\
		(ii) the edges are generated by elements of $\Lambda$,\\
		(iii) generators out of edges form a $\mathbb{Z}$-basis of $\Lambda$. 
	\end{defn}
	
	The toric symplectic manifolds are in bijection with Delzant polytopes.
	
	\begin{eg}
		$(S^1)^n$ acts on $\C^n$ by
		$$ (exp(i\theta_1, \ldots, exp(i\theta_n)).(z_1,\ldots, z_n)=(exp(i\theta_1)z_1, \ldots)$$
		The moment map $\mu: \C^n \mapsto \R^n$
		$$(z_1,\ldots, z_n)\mapsto 1/2(|z_1|^2, \ldots)$$
		The image is the positive quadrant: $\{(x_1,\ldots, x_n) \in \R^n|x_i \geq 0\}$
		If all $z_i$ are nonzero then we get our previous local model $V \times V^{*}/{\Lambda}^{*}$ where V=$(\R^n)^{*}$.\\
		
		In $\R^2$ the generic fibre is $S^1 \times S^1$ with radii $\sqrt{x_1}, \sqrt{x_2}$.
		As $x_2 \to 0$,the fibre collapses down to $S^1$, and similarly when $x_1$ tends to 0.
		
		More invariantly, over a codimension k face of $(\R_{\geq 0})^n \in \R^n$ modelled on $W \subset V$, collapse $W^{\perp} \subset V^* $ to get the local model $ W \times W^{*}/{\Lambda_W}^{*}$.\\
		
		The boundary is stratified by lower dimensional toric varieties.
	\end{eg}
	
	
	
	
	Starting from these local models one can construct a toric variety by glueing these charts. (This is useful if you want to work in coordinates.)
	
	
	
	
	
	\section{Local algebro-geometric construction from the fan}
	N=lattice, say $\Z^{n}$, and let $N_{\R}=N\otimes \R $.
	
	\begin{defn}
		$\sigma \subset N_{\R}$ is called a rational polyhedral cone
		if $\sigma=\{a_1 v_1+a_2 v_2+\ldots+ a_r v_r|a_i \geq 0\}$ for some $v_1, \ldots v_r \in N$ 
	\end{defn}
	
	\begin{defn}
		$\sigma$ is called a strongly convex rational polyhedral cone, if in addition $\sigma \cap -\sigma=0$.  We will refer to these just as cones.
	\end{defn}
	
	
	\begin{defn}
		The dual cone: $\sigma^{\vee} \in M_{\R}$ is $\{f \in M_{R}: f(v)\geq 0 for\: all\: v \in \sigma\}$, where $M=Hom(N,\Z)$ and $M_\R=M\otimes \R$.
	\end{defn}
	
	\begin{eg}
		$N=\Z^2$, $\sigma$=cone with generators (1,0) and (-1,-1), then $M=(\Z^2)^{\vee}$ and $\sigma^\vee$=cone with generators (0,-1) and (1,-1).
	\end{eg}
	
	\begin{eg}
		$N=\Z^2$, $\sigma$=cone with generators (-1,-1), then $M=(\Z^2)^{\vee}$ and $\sigma^\vee$=cone with generators (1,-1) , (-1,1), and (0,-1).
	\end{eg}
	
	Notice inclusion of cones give rise to reverse inclusions of dual cones.
	
	We make the construction: take a cone in N, then set $X_\sigma=Spec \C[\sigma^{\vee}\cap M]$ (\ie,take the semigroup algebra, then take spec).
	
	If $\sigma$ has a face (\ie, the intersection of $\sigma$ with a supporting hyperplane), then $\tau^\vee \supset \sigma^\vee$, so $\C[\sigma^{\vee}\cap M] \hookrightarrow \C[\tau^{\vee}\cap M]$ (localisation of rings), 
	so $X_\tau \hookrightarrow X_\sigma$ an inclusion of princial open affine piece.
	
	\begin{defn}
		A fan $\Sigma \in N$ is a collection of cones $\sigma$, such that:
		\begin{itemize}
			\item if $\sigma \in \Sigma$ and $\tau$ is a face of $\sigma$, then $\tau \in \Sigma$.
			\item if $\sigma, \sigma^{'} \in \Sigma$, then $\sigma \cap \sigma^{'}$ is a face of each.
		\end{itemize}
	\end{defn}
	
	so we can glue the $X_{\sigma}$ along the intersections! 
	Then one defines in a natural manner the torus action to obtain an actual toric variety.
	
	\begin{eg}
		$N=\Z, N_{\R}\simeq \R$, then we have the fan defined by: $\tau$=origin, $\sigma_1$=positive real line, $\sigma_2$=negative real line, then the dual of $\sigma_1$ and $\sigma_2$ correspond to the positive and negative real lines in the dual space. We have: $\C[\sigma_1^{\vee}\cap M]=\C[x] $, $\C[\sigma_2^{\vee}\cap M]=\C[x^{-1}] $, and $\C[\tau^{\vee}\cap M]=group \: algebra=\C[x, x^{-1}] $. Geometrically we have two inclusions of $\C^*$ into $\C$ so we get $\mathbb{P}^1$.
	\end{eg}
	
	\begin{ex}
		Do the same thing for the fan defined by the first quadrant in $\R^2$.
	\end{ex}
	
	\begin{ex}
		Do the same thing for the fan defined by the three rays in the directions (0, 1), (1,0), (-1,-1).
	\end{ex}
	
	
	\begin{ex}
		Do the same thing for the fan defined by the first quadrant in $\R^2$, subdivided along the diagonal.
		
	\end{ex}
	
	
	
	\begin{ex}
		Find out what a map of fans is and show that it induces a morphism. Interprete these maps geometrically in the examples. 
		
		Give a necessary and sufficient condition on  $\Sigma$ for $X_{\Sigma}$ to be smooth.
		Prove one can always desingularise the toric variety in the toric category.
		
	\end{ex}
	
	
\end{document}
